\documentclass[a4paper,12pt]{article}
\usepackage{ dsfont }
\usepackage{fullpage}
\usepackage[ngerman]{babel}
%\AtBeginDocument{\renewcommand{\chaptername}{}}
\usepackage{amsmath}
\usepackage{amssymb}
\usepackage{chngcntr}
%\counterwithout{equation}{chapter}
\usepackage[utf8]{inputenc}
%\usepackage{graphicx}
%\usepackage{listings}
%\usepackage{epstopdf}
\usepackage{caption}
%\usepackage{showidx}
\usepackage{makeidx}
\usepackage{enumerate}	
\usepackage{ifthen}
%\usepackage{ bbold }
\usepackage{ textcomp }
\usepackage{xcolor}
\usepackage{graphicx}
\usepackage{hyperref}
 
\newcommand{\ci}{\perp\!\!\!\perp}
\newenvironment{Bew}{\textbf{Beweis:}}{  \hfill$\square$ \newline}
\newcommand{\ExiF}{\text{$\E(\xi | \F)$}}
\newcommand{\F}{\text{$\mathfrak{F}$}}
\newcommand{\G}{\mathfrak{G}}
\newcommand{\E}{\mathds{E}}
\newcommand{\Prob}{\mathds{P}}

\newcommand{\defi}[1]{\index{#1}
	\index{Definition!#1}
	\textit{#1}}
\newtheorem{Satz}{Satz}	
\newtheorem{Lemma}[Satz]{Lemma}
\newtheorem{Bemerkung}[Satz]{Bemerkung}
\newtheorem{Korrolar}[Satz]{Korollar}
\newtheorem{Definition}[Satz]{Definition}
\newtheorem{Fakta}[Satz]{Fakta}
\newtheorem{Proposition}[Satz]{Proposition}
\newtheorem{Beispiel}[Satz]{Beispiel}
\newcommand{\spindex}[1]{\index{0@#1}}
\newcommand{\eps}{\varepsilon}
\newcommand{\alg}{\mathcal{A}}

\newenvironment{Defi}[2][]{\begin{Definition}
		\index{#1}
		\index{Definition!#1}
		\index{#2}
		\index{Definition!#2}
		\rm
		%#1
	}{\end{Definition}}
\newenvironment{Lem}{\begin{Lemma}	\rm }{\end{Lemma}}
\newenvironment{Bem}{\begin{Bemerkung} \rm}{\end{Bemerkung}}
\newenvironment{Fak}{\begin{Fakta} \rm}{\end{Fakta}}
\newenvironment{Kor}{\begin{Korrolar} \rm}{\end{Korrolar}}
\newenvironment{Sat}{	\begin{Satz} \rm}{ \end{Satz} }
\newenvironment{Bsp}{\begin{Beispiel} \rm}{\end{Beispiel}}
%	\begin{Satz}[] \rm}{\end{Satz}}
\newenvironment{Def}{\begin{Definition}	\rm}{\end{Definition}}
\newenvironment{Pro}{\begin{Proposition} \rm}{\end{Proposition}}
\newcommand{\ES}{\mathcal{S}}
\newcommand{\supp}{\text{supp}}
\newcommand{\one}[1][]{\mathds{1}_{#1}}
\newcommand{\mom}{m}
\newcommand{\mn}[1][]{[m_0,...,m_{2\ifthenelse{\equal{#1}{}}{n}{ #1}}]}
\newcommand{\mne}[1][]{[m_1,...,m_{2\ifthenelse{\equal{#1}{}}{n}{ #1}+1}]}
\newcommand{\lnn}[1][]{[\el_0,...,\el_{\ifthenelse{\equal{#1}{}}{2n}{2 #1}}]}
\newcommand{\lne}[1][]{[\el_1,...,\el_{2\ifthenelse{\equal{#1}{}}{n}{ #1}+1}]}
\newcommand{\vn}[1][]{[\vau_0,...,\vau_{2\ifthenelse{\equal{#1}{}}{n}{ #1}}]}
\newcommand{\vne}[1][]{[\vau_1,...,\vau_{2\ifthenelse{\equal{#1}{}}{n}{ #1}+1}]}
\newcommand{\el}{\lambda}
\newcommand{\vau}{\nu}
\newcommand{\deq}{\stackrel{d}=}
\newcommand{\spann}{\text{span}}

\numberwithin[\arabic]{Satz}{section}
\setlength{\parskip}{2pt}
\setlength{\parindent}{0pt}
\definecolor{Tublau}{HTML}{006699}
\renewcommand{\labelenumi}{(\roman{enumi})}
\begin{document}
	\section*{Rainbow-Options}
	Autoren: Florian Bogner, Hubert Hackl, Martin Vetter, Thomas Wagenhofer
		\subsection*{Grundlegendes}
			Im folgenden werden Rainbow-Options betrachtet. Diese sind im Gegenzug zu konventionellen Optionen nicht notwendigerweise auf nur ein Underlying und einen Strike Preis aufgebaut.
			\newline
			Konkret werden diese Optionen betrachtet:
			\begin{itemize}
				\item Abgabe einer Pepsi-Aktie für 2.3 Anteile einer Coca Cola Aktie
				\item Abgabe von 2.3 Anteile einer Coca Cola Aktie für eine Pepsi Aktie
				\item Dem Kauf des Maximums aus dem Basket zu einem Strikepreis K
				\item Dem Kauf des Minimums aus dem Basket zu einem Strikepreis K
				\item Dem Kauf des Maximums aus dem Basket zum 16.5-fachen Kurs von XLK\footnote{Technology Select Sector SPDR Fund, ein Fond bestehend aus einigen IT-Firmen}
				\item Dem Kauf des Minimums aus dem Basket zum 16.5-fachen Kurs von XLK
			\end{itemize}
			Der genannte Basket besteht dabei aus folgenden Aktien: 6.5 $\cdot$ Apple, 1 $\cdot$ Alphabet Inc., 22.5 $\cdot$ Intel, 9 $\cdot$ IBM und 7.3 $\cdot$ NVIDIA.
			Die Optionen werden jeweils zu den Expirationdates 1.2.2019, 15.2.2019 und 21.6.2019 bewertet.
		\subsection*{Vorgehensweise}
		Zur Optionsbewertung verwenden wir das Hestonmodell, welches wir vermittels Euler-Maruyama Verfahren diskretisieren und implementieren. Die Besonderheit des Hestonmodells liegt daran, dass es eine explizite Formel für Call-Preise existiert. Den endgültigen Optionspreis erhalten wir dann mit einer Monte Carlo Simulation.\\
		Die Anfangsparameter Volatilität $v_0$, mittlere Volatilität $\bar{v}$, Korrelation zwischen Volatilität und Stockprice $\rho$, mean-reversion Parameter $\kappa$ und die Volatilität der Volatilität $\sigma$ ermitteln wir mit einem Gradientenverfahren. Hier geht die Existenz einer geschlossenen Formel für Call-Preise ein. Gegeben seien aktuelle Preise von Call-Optionen $C_i^*$ zu Zeiten $T_i$ mit strike $K_i$. Ziel ist es, einen Parameter $\theta^*$ so zu wählen, dass die quadratischen Fehler minimiert werden:
		\begin{gather*}
		\theta^*=\text{argmin}_{\theta} f(\theta):=\sum_{i=1}^n \frac12 (C(\theta,T_i,K_i)-C^*)^2.
		\end{gather*}
		Durch eine geschlossene Formel, die in dem R-Package \glqq NMOF\grqq\footnote{Basierend auf "Numerical Methods and Optimization in Finance" von M. Gilli, D. Maringer und E. Schumann (2011), ISBN 978-0123756626} implementiert ist, kann man ein effizientes Gradientenverfahren durchführen. Es zeigte sich, dass man für beliebig kleine Fehler ein geeignetes $\theta$ finden kann. Das Ergebnis hängt jedoch von der Wahl des Anfangspunktes ab, da nur lokale Minima gesucht werden. Deshalb ist es ratsam, Ergebnisse noch auf Plausibilität zu überprüfen.
		
		Als Zinssatz $\mu$ verwenden wir die Durchschnittsrendite von zehnjährigen deutschen Staatsanleihen.\\
		Schließlich benötigen wir noch die Korrelationen (welche wir als konstant annehmen!) zwischen den Stocks um mit diesen unsere Kovarianzmatrix für die in den Modellen auftretenden Normalverteilungen aufzustellen.
		Wegen der Identität:
		\begin{align*}
			\rho_{1,2}&=Cor(S_t,\tilde{S_t})=\\
			&=Cor(S_0+\mu S_0h+\sqrt{v_0}S_0\sqrt{h}N_1,\tilde{S_0}+\mu \tilde{S_0}h+\sqrt{v_0}\tilde{S_0}\sqrt{h}N_2)= \\
			&=Cor(\sqrt{v_0}S_0\sqrt{h}N_1,\sqrt{v_0}\tilde{S_0}\sqrt{h}N_2)=Cor(N_1,N_2)
		\end{align*}
		können wir die Korrelationsparameter aus historischen Daten berechnen.
		\newpage
	\subsection*{Coca-Cola/Pepsi Optionen}
		Wir bewerten die Optionen: 
		\begin{itemize}
			\item[i)] Abgabe einer Pepsi-Aktie(PEP) für 2.3 Anteile einer Coca Cola Aktie(KO) mit Payoff-function:
				\begin{equation*}
				P=max((KO-2.3*PEP),0)
				\end{equation*}			
			\item[ii)] Abgabe von 2.3 Anteile einer Coca Cola Aktie für eine Pepsi Aktie mit Payoff-function:
			\begin{equation*}
			P=max((2.3*PEP-KO),0)
			\end{equation*}			
		\end{itemize}
		Die Euler-Maruyama Diskretisierung des Hestonmodells liefert:
		\begin{align*}
		S^{KO}_{t+1}&=S^{KO}_{t}+\mu S^{KO}_{t}h+\sqrt{v^{KO}_t}S^{KO}_{t}\sqrt{h}N_1\\
		v^{KO}_{t+1}&=\kappa^{KO}(\bar{v}^{KO}-v^{KO}_t)h+v^{KO}_t+\sigma^{KO}\sqrt{v^{KO}_t}\sqrt{h}N_2\\
		S^{PEP}_{t+1}&=S^{PEP}_{t}+\mu S^{PEP}_{t}h+\sqrt{v^{PEP}_t}S^{PEP}_{t}\sqrt{h}N_3\\
		v^{PEP}_{t+1}&=\kappa^{PEP}(\bar{v}^{PEP}-v^{PEP}_t)h+v^{PEP}_t+\sigma^{PEP}\sqrt{v^{PEP}_t}\sqrt{h}N_4
		\end{align*}
		Wir benötigen also eine vierdimensionale Normalverteilung mit:
		\begin{equation*}
		\left({\begin{array}{c} N_1\\ N_2\\ N_3\\ N_4\end{array}}\right) \sim N\left(\left({\begin{array}{c} 0 \\ 0 \\ 0 \\ 0 \end{array}}\right),\left({\begin{array}{cccc} 1 & \rho^1_2 & \rho^1_3 & \rho^1_4\\ \rho^1_2 & 1 & \rho^2_3 & \rho^2_4 \\ \rho^1_3 & \rho^2_3 & 1 & \rho^3_4 \\ \rho^1_4 & \rho^2_4 & \rho^3_4 & 1 \end{array}}\right)\right)
		\end{equation*}
		Wobei $\rho^{i}_j$ die Kovarianz von $N_i$ und $N_j$ bezeichnet.
		Die Parameter $\rho^1_2$ und $\rho^3_4$ sind die Kovarianzen zwischen KO-Volatilität und KO-Stockprice (bzw. PEP-Volatilität und PEP-Stockprice) und werden daher vom Gradientenverfahren geliefert. 
		
	\subsection*{Basket Optionen}
	
	
	
	
	
	
	
	
	
	
	
	
	
	
	
	
	
	
	


\begin{verbatim}
Exchange PEP for 2.3*KO:
2019-02-01 2019-02-15 2019-06-21
     4.57$      6.99$     17.94$ 

Exchange 2.3*KO for PEP:
2019-02-01 2019-02-15 2019-06-21
     4.73$      7.02$     15.89$
\end{verbatim}
     
\begin{verbatim}
     Call on Max:
K\T  2019-02-01 2019-02-15 2019-06-21
1050    100.59$    142.39$    333.03$ 
1060     90.94$    133.08$    324.20$ 
1072     79.76$    122.19$    313.76$ 
1120     42.67$     84.02$    273.82$ 

Call on Min:
K\T  2019-02-01 2019-02-15 2019-06-21
1050      1.60$      2.91$      8.71$ 
1060      0.89$      1.99$      7.56$ 
1072      0.37$      1.20$      6.34$ 
1120      0.00$      0.10$      2.79$ 

Exchange with Max:
2019-02-01 2019-02-15 2019-06-21
   109.05$    172.37$    377.73$ 

Exchange with Min:
2019-02-01 2019-02-15 2019-06-21
    11.03$     23.03$     40.62$ 
    \end{verbatim}
	
	
	
		
	
\end{document}